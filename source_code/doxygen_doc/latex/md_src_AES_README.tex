\subsection*{1-\/ A\+ES L\+I\+B\+R\+A\+RY }

In order to avoid conflicts with the G\+PL license of A\+VR Cryptolib we have decided to change the A\+ES Library to \href{http://www.literatecode.com/aes256}{\tt http\+://www.\+literatecode.\+com/aes256}.

To avoid changes in the current C\+TR implementation we decided to do some \#define and avoid changing function names and those things.

Changes made from the original library to avoid changes in other files\+: 
\begin{DoxyCode}
1 aes.c:
2 
3 1.Uncomment '#define BACK\_TO\_TABLES'
4 
5 2.Add sbox and sboxinv to flash memory:
6 const uint8\_t sbox[256] \_\_attribute\_\_ ((\_\_progmem\_\_)) = \{...\};
7 const uint8\_t sboxinv[256] \_\_attribute\_\_ ((\_\_progmem\_\_)) = \{...\};
8 
9 3.Modify #define of rj\_sbox and rj\_sbox\_inv to:
10 #define rj\_sbox(x)     (pgm\_read\_byte(&sbox[x]))
11 #define rj\_sbox\_inv(x) (pgm\_read\_byte(&sboxinv[x]))
12 
13 4.Change aes\_init function name to aes256\_init\_ecb
\end{DoxyCode}
 
\begin{DoxyCode}
1 aes.h:
2 
3 1- Add '#define' inside the header file
4 #define aes256\_ctx\_t aes256\_context
5 
6 #define aes256\_init(x,y)    aes256\_init\_ecb((y),(uint8\_t*)(x))
7 #define aes256\_enc(x,y)     aes256\_encrypt\_ecb((y),(uint8\_t*)(x))
8 #define aes256\_dec(x,y)     aes256\_decrypt\_ecb((y),(uint8\_t*)(x))
\end{DoxyCode}


How to use the library? How to work with it ? As easy as it sounds, you only have to care about 3 functions\+: aes256\+\_\+init, aes256\+\_\+enc and aes256\+\_\+dec. Here it is a simple example\+:


\begin{DoxyCode}
1 void aes256\_test(void)
2 \{
3     // aes256 is 32 byte key
4     uint8\_t key[32] = \{0,1,2,3,4,5,6,7,8,9,10,11,12,13,14,15,16,17,18,19,20,
       21,22,23,24,25,26,27,28,29,30,31\};
5 
6     // aes256 is 16 byte data size
7     uint8\_t data[16] = "text to encrypt";
8 
9     // declare the context where the round keys are stored
10     aes256\_ctx\_t ctx;
11 
12     // Initialize the AES256 with the desired key
13     aes256\_init(key, &ctx);
14 
15     // Encrypt data
16     // "text to encrypt" (ascii) -> '9798D10A63E4E167122C4C07AF49C3A9'(hex)
17     aes256\_enc(data, &ctx);
18 
19     // Decrypt data
20     // '9798D10A63E4E167122C4C07AF49C3A9'(hex) -> "text to encrypt" (ascii)
21     aes256\_dec(data, &ctx);
22 \}
\end{DoxyCode}


\subsection*{2-\/ Testing the library using nessie test vectors }

After downloading a third party library or resource you must ensure the library performs the function as well as it is claimed. So to satisfy our paranoia against any bug or error with the library, we have checked the encryption and decryption using different test vectors, called Nessie Test Vectors. There are 8 different sets of test vectors, we have checked A\+E\+S256 against all.

To test A\+E\+S256 using nessie vectors, we have created a file called \hyperlink{aes256__nessie__test_8c}{aes256\+\_\+nessie\+\_\+test.\+c}. This file outputs the results of nessie test into U\+A\+RT, U\+SB C\+DC or whatever function you want. You only have to initialize the function pointer to print the output where you want.

Sample code to print the output through U\+SB C\+DC\+:


\begin{DoxyCode}
1 #include "aes256\_nessie\_test.h"
2 
3 void main(void)
4 \{
5     /*
6         INITIALIZATION OF USB CDC
7     */
8 
9     // Redirect nessieOutput to usb\_serial\_putchar
10     nessieOutput = &usb\_serial\_putchar;
11 
12     // Test all sets of nessie vectors
13     nessieTest(1);
14     nessieTest(2);
15     nessieTest(3);
16     nessieTest(4);
17     nessieTest(5);
18     nessieTest(6);
19     nessieTest(7);
20     nessieTest(8);
21 \}
\end{DoxyCode}


Nessie test vectors and output are located in \href{https://www.cosic.esat.kuleuven.be/nessie/testvectors}{\tt https\+://www.\+cosic.\+esat.\+kuleuven.\+be/nessie/testvectors} Block Cipher -\/$>$ Rijndael -\/$>$ key size 256.

The output of all nessie\+Test functions are formatted in the same way as the file {\bfseries aes256\+\_\+nessie\+\_\+test.\+txt}, so... you must save the output (using cutecom or similar hyperterminal program) into a file and check the differences between your file and {\bfseries aes256\+\_\+nessie\+\_\+test.\+txt} using a diff viewer.

\subsection*{3-\/ C\+TR block encryption }

The passwords stored on the mooltipass will be encrypted using C\+TR block encryption, more information in\+: \href{http://en.wikipedia.org/wiki/Block_cipher_mode_of_operation#Counter_.28CTR.29}{\tt Counter C\+TR }. We must decide how to generate the initialization vector. Here\textquotesingle{}s an example of use of C\+TR encryption and decryption.


\begin{DoxyCode}
1 static uint8\_t key[32] = \{ 0x60, 0x3d, 0xeb, 0x10, 0x15, 0xca, 0x71, 0xbe, 0x2b,
2 0x73, 0xae, 0xf0, 0x85, 0x7d, 0x77, 0x81, 0x1f, 0x35, 0x2c, 0x07, 0x3b,
3 0x61, 0x08, 0xd7, 0x2d, 0x98, 0x10, 0xa3, 0x09, 0x14, 0xdf, 0xf4 \};
4 
5 static uint8\_t iv[16] = \{ 0xf0, 0xf1, 0xf2, 0xf3, 0xf4, 0xf5, 0xf6, 0xf7, 0xf8,
6 0xf9, 0xfa, 0xfb, 0xfc, 0xfd, 0xfe, 0xff \};
7 
8 
9 char text[32] = "this is my pass to encrypt";
10 
11 void main(void)
12 \{
13     /*
14         Stuff here
15     */
16 
17     // Declare aes256 context variable
18     aes256CtrCtx\_t ctx;
19 
20     // Save key and initialization vector inside context
21     aes256CtrInit(&ctx, key, iv, 16);
22 
23     aes256CtrEncrypt(&ctx, text, sizeof(text));
24     // here array pass has been encrypted inside text
25 
26     /*
27         Decrypt
28     */
29 
30     aes256CtrDecrypt(&ctx, text, sizeof(text));
31     // decrypting make text to be "this is my pass to encrypt" again.
32 \}
\end{DoxyCode}
 Data input in aes256\+Ctr\+Encrypt and aes256\+Ctr\+Decrypt must be multiple of 16 bytes length.

As said, we like to test our code and verify it has no bugs so... you can test C\+TR encryption implementation by using {\bfseries \hyperlink{aes256__ctr__test_8c}{aes256\+\_\+ctr\+\_\+test.\+c}} functions and doing a diff against {\bfseries aes256\+\_\+ctr\+\_\+test.\+txt}. C\+TR vectors used in aes256\+Ctr\+Test are from \href{http://csrc.nist.gov/publications/nistpubs/800-38a/sp800-38a.pdf}{\tt http\+://csrc.\+nist.\+gov/publications/nistpubs/800-\/38a/sp800-\/38a.\+pdf}. 
\begin{DoxyCode}
1 void main(void)
2 \{
3 
4     /*
5         INITIALIZATION OF USB CDC
6     */
7 
8     // Redirect ctrTestOutput to usb\_serial\_putchar function
9     ctrTestOutput = &usb\_serial\_putchar;
10 
11     aes256CtrTest();
12 \}
\end{DoxyCode}


\subsection*{4-\/ Description of files }


\begin{DoxyItemize}
\item A\+V\+R-\/cryptolib files used in this project\+:
\end{DoxyItemize}


\begin{DoxyCode}
1 aes.c -> AES256 implementation from http://www.literatecode.com/aes256 (with some things changed)
\end{DoxyCode}



\begin{DoxyItemize}
\item Custom files done by mooltipass team\+:
\end{DoxyItemize}


\begin{DoxyCode}
1 aes256\_nessie\_test.c    (only used for test)
2 aes256\_ctr\_test.c       (only used for test)
3 aes256\_ctr.c
\end{DoxyCode}



\begin{DoxyItemize}
\item Files to check aes256\+\_\+nessie and aes256\+\_\+ctr tests\+:
\end{DoxyItemize}


\begin{DoxyCode}
1 aes256\_nessie\_test.txt
2 aes256\_ctr\_test.txt
\end{DoxyCode}


\subsection*{5-\/ Speed performance }


\begin{DoxyCode}
1 with #define BACK\_TO\_TABLES enabled
2 text     data     bss     dec    hex filename
3 1872       0       0    1872     750 aes.o
4 
5 Time(1000 encryptions): 59433 ms
6 
7 
8 with #define BACK\_TO\_TABLES disabled
9 text    data     bss     dec     hex filename
10 2320       0       0    2320     910 aes.o
11 
12 Time(1000 encryptions): 1204 ms
\end{DoxyCode}
 